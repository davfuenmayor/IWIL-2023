\documentclass{article}
\usepackage{graphicx} % Required for inserting images

\title{Searching for Wormholes in Proof-Space}
\author{Christoph Benzmüller and David Fuenmayor}
\date{University of Bamberg and Freie Universität Berlin}

\begin{document}

\maketitle

\section{Introduction}

Workshop webpage: \url{https://eprover.org/EVENTS/IWIL-2023.html}
\\
\textbf{Motivating text:}
\\

\textbf{From ChatGPT:}
\\
"Proof-space is a fascinating concept in mathematics that describes the vast and intricate space of all possible mathematical proofs. This space is filled with countless paths, each leading to a different solution and each one of them unique in its own way. To navigate this complex space, mathematicians often use lemmas, small intermediate results that are proven separately and then used as building blocks for larger proofs.

One of the most intriguing features of proof-space is the existence of wormholes. These are shortcuts or connections between seemingly distant parts of proof-space, and lemmas often play a crucial role in uncovering these connections. By proving a lemma that connects two seemingly unrelated parts of mathematics, mathematicians can open up new wormholes that allow for breakthroughs and insights that were previously thought impossible.

Wormholes in proof-space can also allow for unexpected solutions to long-standing mathematical problems. By using lemmas to connect different parts of proof-space, mathematicians can find entirely new approaches to solving problems that had previously seemed insurmountable.

The discovery of a new wormhole in proof-space can be a thrilling moment for mathematicians, and often requires a deep understanding of the underlying concepts and the ability to recognize patterns and connections. By using lemmas as tools to uncover these connections, mathematicians can push the boundaries of what is possible in mathematics and make breakthroughs that have real-world applications.

In conclusion, proof-space is a complex and fascinating realm in which lemmas play a crucial role in navigating the space and uncovering new wormholes. By proving lemmas that connect seemingly unrelated parts of mathematics, mathematicians can find new approaches to solving problems and make groundbreaking discoveries that have far-reaching implications. As we continue to explore this vast and intricate space, we will undoubtedly continue to rely on lemmas as tools for uncovering new connections and pushing the boundaries of what is possible in mathematics."
\\
\textbf{From the foreword (by Dolph Ulrich) of the book "Automated Reasoning and the Discovery of Missing and Elegant Proofs" by Larry Wos \& Gail W. Pieper}
\\
"[...] The overriding difficulty met at every turn was the unimaginably vast size of the space of proofs, a space in which all proofs solving a particular problem at hand might well be as unreachable as the farthest stars in the most distant galaxies. Consideration of quite short proofs suffices to illustrate this combinatorial explosion: even for systems of logic of the sort studied in this book that have just one axiom, for instance, there can be more 10-step proofs than kilometers in a light year, more 15-step proofs than stars in a trillion Milky Ways.
In the face of such examples, one can hardly fail to ask if the very idea of discovering methods for finding proofs of theorems whose shortest demonstrations might be 25, 50, or 100 steps in length is simply an impossible dream. Wos’s own conviction (high-stakes poker player that he also happened to be) was that the answer is “No” and that the key to fruitfully exploring such immense search spaces lay—perhaps his most important insight of all—in the employment of strategy. In consequence, he made it one of his personal quests to discover a substantial set of particular strategies that could be drawn on and used, in varying combinations, on future problems of all types. [...]"


\section{A Motivating Example: Boolos' Curious Inference}

TODO: from Isabelle/HOL sources (on-demand)

\section{Further Examples}
TODO: from Isabelle/HOL sources (on-demand)

\section{Conclusion}

\end{document}
