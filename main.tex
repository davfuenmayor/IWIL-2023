\documentclass{article}
\usepackage{graphicx,url} % Required for inserting images
\usepackage[margin=1.5in]{geometry}

\title{Searching for Wormholes in Proof-Space}
\author{David Fuenmayor \& Christoph Benzmüller}
\date{\normalsize Otto-Friedrich-Universität Bamberg and Freie Universität Berlin}

\begin{document}

\maketitle

\section{Motivation}

The present `position paper' for presentation at the IWIL-2023 workshop aims at further deepening the discussion initiated in \cite{LostProof} and recently expanded in \cite{BCPpaper}, regarding the intrinsic advantages of reasoning in higher-order logics and the role of `cut' lemmata and definitions in finding `too-long-for-cut-avoiding-first-order-reasoners' proofs.
The leitmotif of this work can thus be suggestively summarized as:
$$\textit{definitions and cut-lemmata can behave like wormholes in proof-space},$$
where the proof-space analogy has been charmingly illustrated in the foreword (by Dolph Ulrich) of the book ``Automated Reasoning and the Discovery of Missing and Elegant Proofs'' by Larry Wos \& Gail W. Pieper \cite{wos2003automated}. We quote: 
\\ \\
\textit{``The overriding difficulty met at every turn was the unimaginably vast size of the space of proofs, a space in which all proofs solving a particular problem at hand might well be as unreachable as the farthest stars in the most distant galaxies. Consideration of quite short proofs suffices to illustrate this combinatorial explosion: even for systems of logic of the sort studied in this book that have just one axiom, for instance, there can be more 10-step proofs than kilometers in a light year, more 15-step proofs than stars in a trillion Milky Ways.''}
\\
%"In the face of such examples, one can hardly fail to ask if the very idea of discovering methods for finding proofs of theorems whose shortest demonstrations might be 25, 50, or 100 steps in length is simply an impossible dream. Wos’s own conviction (high-stakes poker player that he also happened to be) was that the answer is “No” and that the key to fruitfully exploring such immense search spaces lay—perhaps his most important insight of all—in the employment of strategy. In consequence, he made it one of his personal quests to discover a substantial set of particular strategies that could be drawn on and used, in varying combinations, on future problems of all types. [...]"}

In previous work \cite{BCPpaper} a thought experiment was presented as a motivating example: Folbert and Holly (waiting at the gates of heaven) become engaged in a theorem proving contest in which they have to pose first-order proof problems to each other, and the one whose ATP solves the given problem the faster will be admitted to heaven. Folbert goes for first-order (FO) ATPs and Holly for higher-order ATPs. We quote:
\\ \\
\textit{``Key to Holly's advantage are the (hyper-)exponentially shorter proofs that are possible as one moves up the ladder of expressiveness from first-order logic to second-order logic, to third-order logic, and so on  \cite{GoedelProofLength}. The fact that the proof problems are stated in FO logic does not matter. When stating the same problem in the same FO way but in higher-order logic, much shorter proofs are possible, some of which might even be (hyper-)exponentially shorter than the proofs that can be found with comparatively inexpressive FO ATPs. A very prominent example of such a short proof is that of \emph{Boolos' Curious Inference} \cite{BCI}.}"
\\ \\
In this presentation we will discuss three examples of what we see as `wormholes' in proof-space. The first one draws upon \emph{Boolos' Curious Inference} \cite{BCI}, as discussed in \cite{BCPpaper}, and has a motivational character; it serves to illustrate the astronomical magnitudes involved. The second one concerns the area of \textit{correspondence theory} for modal logics. A third one, which is still work-in-progress, concerns applications in formalized mathematics, and involves the notion of compactness in topology (and to some extent also in logic).


\section{Case Studies}
We encode and analyze our case studies using the proof assistant Isabelle/HOL \cite{Isabelle}, which we employ mainly as a frontend (with sophisticated editor/syntax capabilities) in order to invoke concurrently many different first- and higher-order backend ATPs, via the integrated meta-prover Sledgehammer \cite{blanchette2016hammering}. The encoding has been carried out using `vanilla' simple type theory, avoiding library definitions and lemmata as much as possible (theory `Main' is imported for technical reasons). No Isabelle-specific extensions (locales, type classes, etc.) have been employed. The idea is that our proof problems shall remain self-contained and as close as possible to \texttt{THF} syntax \cite{THF}, so our experiments can be easily replicated directly on the concerned provers (as done in \cite{BCPpaper}).

The corresponding Isabelle/HOL source files can be consulted under:\footnote{Because of space and time constraints we cannot discuss here the contents of these case studies. We encourage the reader to consult the provided source files which have been carefully commented and --so we believe-- or more pedagogic value as an interactive document than as \LaTeX\ prose.} \\ \url{https://github.com/davfuenmayor/IWIL-2023/tree/main/sources}

\paragraph*{Boolos' Curious Inference (BCI)}
We encode the original BCI problem (file \texttt{BCI.thy}) employing algebraic notions and discuss the role of cut-definitions in enabling fully automatic proofs. Subsequently, we present a natural algebraic generalization of BCI (file \texttt{BCIgen.thy}) and show how this can also be solved automatically by ATPs. We also show that this general `cut-lemma' can be automatically employed by ATPs to solve BCI-like problems (e.g.~involving fast-growing Ackermann-like functions).

\paragraph*{Modal Correspondence Theory}
We present an algebraically-flavoured theory (cf.~file \texttt{modal-correspondence.thy}) allowing for automatically proving correspondences between axiomatic conditions on relations and their respective modal operations (`box' and `diamond'). This theory is much less ambitious than others in the modal correspondence market (e.g.~\cite{Conradie2012} and references therein) but is simple enough to nicely illustrate how ATPs can cleverly exploit cut-lemmas to reason in modal and non-classical contexts.

\paragraph*{Compactness in Topology}
We encode the notion of a compact set wrt.~a topology (file \texttt{topology-compactness.thy}) employing definitions of different complexities. We then use ATPs to show under which circumstances those definitions are equivalent. In order to illustrate the prospects of introducing ATPs in formalized mathematics, we explore several `cut-lemmata' that should allow us to prove, fully automatically, a common classroom example, namely that the continuous image of a compact set is compact.\footnote{This is, however, still work-in-progress (and part of a larger project, cf.~\cite{CICM22}).}
\\ 



%\section{Conclusion}

%Workshop webpage: \url{https://eprover.org/EVENTS/IWIL-2023.html}






\bibliographystyle{alpha} 
\bibliography{main}

\end{document}
